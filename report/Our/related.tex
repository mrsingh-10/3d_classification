% !TEX root = template.tex

\section{Related Work}
\label{sec:related_work}

In recent times, object recognition has gained significant attention with the advancements in deep learning and computer vision technologies. These developments led to remarkable progress in the ability of computer systems to identify and recognize objects within images and videos. Many 3D Objects datasets are becoming free and accessible; Different deep learning models have been trained and tested on vast amounts of data and have exhibited good learning and generalization capabilities, surpassing the performance of traditional computer vision approaches. The generalization of the learned feature implies more capability to recognize an object even if it is rotated or is seen with a different angle, a paper that talks about this and create a model that may recognize an object regarding its orientation is "Orientation-boosted Voxel Nets for 3D Object Recognition" \cite{sedaghat2017orientationboosted}, they used pre-trained Convolutional Neural network from the VoxNet \cite{7353481} and re-trained to classify in addition also the rotations of the sample, this showed a clear improvement in the overall accuracy, going from 92\% to 93.9 on the ModelNet10 dataset.\\

In the corresponding field of 3D Object Detection, to overcome one of the challenges that may arise when using grid structures like voxels, namely capturing large context information in the structure itself.
In general, voxel grid solutions cannot capture large context information, which may be crucial for 3D object recognition and therefore classification. For large objects or particular objects, this may result in low accuracy.
State-of-the-art techniques in this field uses Transformer-based and Sparse CNN architectures and one of the relevant paper is Voxel Transformer for 3D Object Detection (VoTr) \cite{mao2021voxel}.
VoTr achieves cutting edge results by implementing both local and dilated attention mechanisms to capture both short and long-range context information by combining submanifold and Sparse modules.

These results can be replicated also in the 3D Object Classification field by integrating 3D CNN backbones like VoTr right before a Classification Network; Thinking about future works, also our model may benefit from incorporating techniques that are useful for capturing large amounts of context information as well.


Instead, with this paper, we propose different way to make the neural networks more independent and self continuous of the information on locality through Bagging and 3D cut-out techniques, and we will show an improvement on the overall accuracy on the task.